%  LaTeX support: latex@mdpi.com 
%  In case you need support, please attach all files that are necessary for compiling as well as the log file, and specify the details of your LaTeX setup (which operating system and LaTeX version / tools you are using).

%=================================================================
\documentclass[hydrology,article,submit,moreauthors,pdftex]{Definitions/mdpi} 

% If you would like to post an early version of this manuscript as a preprint, you may use preprint as the journal and change 'submit' to 'accept'. The document class line would be, e.g., \documentclass[preprints,article,accept,moreauthors,pdftex]{mdpi}. This is especially recommended for submission to arXiv, where line numbers should be removed before posting. For preprints.org, the editorial staff will make this change immediately prior to posting.

%--------------------
% Class Options:
%--------------------
%----------
% journal
%----------
% Choose between the following MDPI journals:
% hydrology

%---------
% article
%---------
% The default type of manuscript is "article", but can be replaced by: 
% abstract, addendum, article, benchmark, book, bookreview, briefreport, casereport, changes, comment, commentary, communication, conceptpaper, conferenceproceedings, correction, conferencereport, expressionofconcern, extendedabstract, meetingreport, creative, datadescriptor, discussion, editorial, essay, erratum, hypothesis, interestingimages, letter, meetingreport, newbookreceived, obituary, opinion, projectreport, reply, retraction, review, perspective, protocol, shortnote, supfile, technicalnote, viewpoint
% supfile = supplementary materials

%----------
% submit
%----------
% The class option "submit" will be changed to "accept" by the Editorial Office when the paper is accepted. This will only make changes to the frontpage (e.g., the logo of the journal will get visible), the headings, and the copyright information. Also, line numbering will be removed. Journal info and pagination for accepted papers will also be assigned by the Editorial Office.

%------------------
% moreauthors
%------------------
% If there is only one author the class option oneauthor should be used. Otherwise use the class option moreauthors.

%---------
% pdftex
%---------
% The option pdftex is for use with pdfLaTeX. If eps figures are used, remove the option pdftex and use LaTeX and dvi2pdf.

%=================================================================
\firstpage{1} 
\makeatletter 
\setcounter{page}{\@firstpage} 
\makeatother
\pubvolume{xx}
\issuenum{1}
\articlenumber{5}
\pubyear{2019}
\copyrightyear{2019}
%\externaleditor{Academic Editor: name}
\history{Received: date; Accepted: date; Published: date}
%\updates{yes} % If there is an update available, un-comment this line

%% MDPI internal command: uncomment if new journal that already uses continuous page numbers 
%\continuouspages{yes}

%------------------------------------------------------------------
% The following line should be uncommented if the LaTeX file is uploaded to arXiv.org
%\pdfoutput=1

%=================================================================
% Add packages and commands here. The following packages are loaded in our class file: fontenc, calc, indentfirst, fancyhdr, graphicx, lastpage, ifthen, lineno, float, amsmath, setspace, enumitem, mathpazo, booktabs, titlesec, etoolbox, amsthm, hyphenat, natbib, hyperref, footmisc, geometry, caption, url, mdframed, tabto, soul, multirow, microtype, tikz

%=================================================================
%% Please use the following mathematics environments: Theorem, Lemma, Corollary, Proposition, Characterization, Property, Problem, Example, ExamplesandDefinitions, Hypothesis, Remark, Definition, Notation, Assumption
%% For proofs, please use the proof environment (the amsthm package is loaded by the MDPI class).

%=================================================================
% Full title of the paper (Capitalized)
\Title{Potential evapotranspiration of managed grasslands - a climate change lysimeter study}

% Author Orchid ID: enter ID or remove command
\newcommand{\orcidauthorA}{0000-0000-000-000X} % Add \orcidA{} behind the author's name
%\newcommand{\orcidauthorB}{0000-0000-000-000X} % Add \orcidB{} behind the author's name

% Authors, for the paper (add full first names)
\Author{Matevž LVremec $^{1,\dagger,\ddagger}$\orcidA{}, Veronika Forstner $^{1,,\dagger\ddagger}$ and Steffen Birk $^{2,\dagger,}$*}

% Authors, for metadata in PDF
\AuthorNames{Firstname Lastname, Firstname Lastname and Firstname Lastname}

% Affiliations / Addresses (Add [1] after \address if there is only one affiliation.)
\address{%
$^{1}$ \quad Affiliation 1; matevz.vremec@uni-graz.at\\
$^{2}$ \quad Affiliation 2; e-mail@e-mail.com}

% Contact information of the corresponding author
\corres{Correspondence: matevz.vremec@uni-graz.at}

% Current address and/or shared authorship
\firstnote{Current address: Heinrich Strasse 26, Instutue for Earth Sciences, University of Graz} 
\secondnote{These authors contributed equally to this work.}
% The commands \thirdnote{} till \eighthnote{} are available for further notes

%\simplesumm{} % Simple summary

%\conference{} % An extended version of a conference paper

% Abstract (Do not insert blank lines, i.e. \\) 
\abstract{\newline
	1. Evaluate default PET models \\
	2. Evaluate calibrated PET models \\
	3. Sensitivity analysis and model uncertainty of calibrated PET models \\
	4. Implication of elevated $CO_2$ in PET models \\
	5. Implication of warming in PET models \\
	6. Combined effect of warming and elevated $CO_2$ in PET models \\
	7. Evaluation, Sensitivity analysis and model uncertainty of calibrated PET models \\
	\cite{white2016python}, \cite{tennoe2018uncertainpy}, \cite{van2015pyideas}, \cite{herman2017salib}, \cite{pianosi2016sensitivity}, \cite{wiecki2013hddm}, \cite{miranda2018pyswarms} \\
	8. Conclusion}

% Keywords
\keyword{keyword 1; keyword 2; keyword 3 (list three to ten pertinent keywords specific to the article, yet reasonably common within the subject discipline.)}

% The fields PACS, MSC, and JEL may be left empty or commented out if not applicable
%\PACS{J0101}
%\MSC{}
%\JEL{}

%%%%%%%%%%%%%%%%%%%%%%%%%%%%%%%%%%%%%%%%%%
% Only for the journal Diversity
%\LSID{\url{http://}}

%%%%%%%%%%%%%%%%%%%%%%%%%%%%%%%%%%%%%%%%%%
% Only for the journal Applied Sciences:
%\featuredapplication{Authors are encouraged to provide a concise description of the specific application or a potential application of the work. This section is not mandatory.}
%%%%%%%%%%%%%%%%%%%%%%%%%%%%%%%%%%%%%%%%%%

%%%%%%%%%%%%%%%%%%%%%%%%%%%%%%%%%%%%%%%%%%
% Only for the journal Data:
%\dataset{DOI number or link to the deposited data set in cases where the data set is published or set to be published separately. If the data set is submitted and will be published as a supplement to this paper in the journal Data, this field will be filled by the editors of the journal. In this case, please make sure to submit the data set as a supplement when entering your manuscript into our manuscript editorial system.}

%\datasetlicense{license under which the data set is made available (CC0, CC-BY, CC-BY-SA, CC-BY-NC, etc.)}

%%%%%%%%%%%%%%%%%%%%%%%%%%%%%%%%%%%%%%%%%%
% Only for the journal Toxins
%\keycontribution{The breakthroughs or highlights of the manuscript. Authors can write one or two sentences to describe the most important part of the paper.}

%\setcounter{secnumdepth}{4}
%%%%%%%%%%%%%%%%%%%%%%%%%%%%%%%%%%%%%%%%%%
\begin{document}
%%%%%%%%%%%%%%%%%%%%%%%%%%%%%%%%%%%%%%%%%%

%%%%%%%%%%%%%%%%%%%%%%%%%%%%%%%%%%%%%%%%%%
\setcounter{section}{-1} %% Remove this when starting to work on the template.
\section{How to Use this Template}
The template details the sections that can be used in a manuscript. Note that the order and names of article sections may differ from the requirements of the journal (e.g., the positioning of the Materials and Methods section). Please check the instructions for authors page of the journal to verify the correct order and names. For any questions, please contact the editorial office of the journal or support@mdpi.com. For LaTeX related questions please contact latex@mdpi.com.
%The order of the section titles is: Introduction, Materials and Methods, Results, Discussion, Conclusions for these journals: aerospace,algorithms,antibodies,antioxidants,atmosphere,axioms,biomedicines,carbon,crystals,designs,diagnostics,environments,fermentation,fluids,forests,fractalfract,informatics,information,inventions,jfmk,jrfm,lubricants,neonatalscreening,neuroglia,particles,pharmaceutics,polymers,processes,technologies,viruses,vision

\section{Introduction}
\cite{obojes2015vegetation} - higher PET with managed grasslands and \cite{leitinger2015impact}


\section{Evaluation of PET models(Materials and methos)}

\subsection{Previous research}
Sensitivity to erros in potential evapotranspiration input \cite{parmele1972errors}
\cite{parmele1974comparisons} and \cite{mcguinness1972comparison}(z0 = 1 for meadow grass)(details of computation)(Blaney-Criddle-only mean T, Jensen-Hasie \cite{jensen1963estimating}-solar radiation with T) compared 9 PET methods using onsite meteo data
(PEnman with empirical wind function applicable to a short grass with rougness coef of 1cm): 
\cite{andreassian2004impact}-sensitivty of PEnman
\begin{equation}
f(u)=0.35+0.0035*u_2
\end{equation}

Search for best PE formula \cite{jensen1990evapotranspiration}, \cite{xu2002cross}

\subsection{Used PET models}
\cite{xu2002cross} \\
Priestley-Taylor \cite{asce2005asce}
\begin{equation}
E = \frac{\alpha}{\lambda} \frac{\Delta}{\Delta +\gamma}R_{n}
\end{equation}
$\alpha$ = 1.26
\\
Makkink (default) \cite{asce2005asce} \cite{makkink1957testing}
\begin{equation}
ET = 0.61 \frac{\Delta}{\Delta+\gamma} \frac{R_s}{\lambda} -0.12 
\end{equation}
\\
Blaney-Criddle \cite{asce2005asce} \\
\begin{equation}
ET = kp (0.46T_a+8.13) 
\end{equation}
where p = percentage of total daytime hours for the used period (daily or
monthly) out of total daytime hours of the year (365x12);k = monthly consumptive use coefficient, depending on vegetation
type, location and season and for the growing season (May to October),
k varies from 0.5 for orange tree to 1.2 for dense natural
vegetation.
Following the recommendation of Blaney and Criddle (1950)\cite{blaney1950determining}, in the first stage of
the comparative study, values of 0.85 and 0.45 were used for the growing season
(April to September) and the non-growing season (October toMarch), respectively.
\\
Hargreaves \cite{hargreaves1975moisture}\cite{hargreaves1985reference}\cite{hargreaves1982estimating}
\begin{equation}
ET = a(T_{max}-T_{min})^{0.5}(T_{mean}+17.8) \frac{R_a}{\lambda \rho _w}
\end{equation}
a=0.0023 \\
Ra is average daily exoatmospheric radiation(extra terrestrial) \\
\\
FAO PM for reference(potential) evapotranspiration \cite{allen1998crop}
\begin{equation}
ET = \frac{0.408\Delta (R_{n}-G) + \gamma \frac{900}{T_a + 273} U_2 (e_{s}-e_{a})}{\Delta +\gamma (1+0.34u_2)}
\end{equation}
\\
Penman \\
\begin{equation}
E = \frac{1}{\lambda \rho_w} \frac{\Delta (R_{n}-G) + K_u \gamma(a_w+b_wu_2)(e_{s}-e_{a})}{\Delta +\gamma}
\end{equation}
where $a_w$ and $b_w$ are wind function coefficients that are usually receive a local or regional calibration.
Parameter $K_u$ =6.43 for ET in mmd$^{-1}$ and $K_u$ =0.268 for ET in mmhour$^{-1}$.
Penman \cite{penman1963vegetation} used for clipped grass \cite{asce2005asce} $a_w$ = 1.0 and $b_w$ = 0.537, respectively,for wind speed in ms $^{-1}$, $es-ea$ in kPa and grass ETo in mmd$^{-1}$. The equations were intended for use with daily computations. In application of the 1963 Penman, saturation vapor pressure is traditionally based on mean daily air temperature rather than on Tmax and Tmin$"$ \cite{jensen1990evapotranspiration}. \\
\\
Kimberley-Penman \cite{asce2005asce}
\begin{equation}
E = \frac{1}{\lambda \rho_w} \frac{\Delta (R_{n}-G) + K_u \gamma(a_w+b_wu_2)(e_{s}-e_{a})}{\Delta +\gamma} 
\end{equation}
where:
\begin{equation}
a_w = 0.4 + 1.4 exp{-\left[\left(\frac{J-173}{58}\right)^2\right]}
\end{equation}
\begin{equation}
b_w = 0.605 + 0.345 exp{-\left[\left(\frac{J-243}{80}\right)^2\right]}
\end{equation}
Reference ET FAO with kc for grass\\
\\
Jensen-Haise \cite{asce2005asce} - ET$_r$ is alfalfa reference ET
\begin{equation}
ET_r = \frac{1}{\lambda} C_r(T-Tx)R_s
\end{equation}
$C_r$ and $T_x$ should be constant for a given area...
Later Jensen defined:
"Jensen and Haise (1963) evaluated 3,000 observations of ET as determined by soil sampling procedures over a 35-year period. From about 100 values for well-watered crops with full cover in the western United States, a linear relationship of a solar radiation coefficient and mean air temperature was apparent. From these data, the constants for the following linear equation were CT=0.025 and Tx=-3 for temperature in C. Rs has the same units as $\lambda$ ETr."
\begin{equation}
C_1 = 38-(2Elev/305)
\end{equation}
\begin{equation}
T_x = -2.5 - 1.4 (e_2-e_1) - Elev/550
\end{equation}
\\
Penman-Monteith \cite{asce2005asce}
\begin{equation}
E = \frac{1}{\lambda} \left[ \frac{\Delta (R_{n}-G)+ K_min \rho_a c_p  (e_{s}-e_{a})/r_{ah}}{\left[ \Delta +\gamma (1+\frac{r_s}{r_{ah}})\right]}\right]
\end{equation}
, where $K_min$ units conversion, equal to 86,400 s d-1 for ET in mm d-1 and equal to 3600 s h-1 for ET in mm h-1.
\begin{equation}
r_a = \frac{ln \left[\frac{z_w-d}{z_{om}}\right]ln \left[\frac{z_h-d}{z_{oh}}\right]}{k^2 u_z}
\end{equation}
where zw = 2, zh = 2, d = 0.67 * h, zom = 0.123*h, zoh = 0.0123*h, k=0.41, u=wind, h=vegetation height
\begin{equation}
r_s = \frac{rl}{LAI_{active}}
\end{equation}
where rl = bulk stomatal resistance of a well-illuminated leaf 100sm-1, and $LAI_{active}$=0.5 LAI. rs=1/gs
\\
Oudin \cite{oudin2005potential}
\begin{equation}
ET = \frac{R_e}{\lambda \rho} \frac{T_a+5}{100}, if T_a + 5 > 0, else ET = 0
\end{equation}
\\
Hamon \cite{oudin2005potential}
\begin{equation}
ET = \left(\frac{DL}{12}\right)^2 exp \left(\frac{T_a}{16}\right)
\end{equation}
\section{Evaluation of PET models with original constant values}
Priestley-Taylor \\
Makkink \\
Blaney-Criddle \\
Hargreaves \\
FAO-pm-reference \\
Kimberley-Penman \\
Jensen-Haise \\
Penman-Monteith \\
Penman \\
Oudin \\
Hamon \\
FAO-pm-reference + crop coefficient \\

\section{Evaluation of PET models with calibrated constant values}

Priestley-Taylor \\
$\alpha$
\\
Makink to calibrate \cite{asce2005asce}
\begin{equation}
ET = f * \frac{1}{\lambda \rho} \left(\frac{0.63 R_s \Delta}{\Delta+\gamma}-14 \right)
\end{equation}
\\
Blaney-Criddle \\
1. k \\
2. a, b 
Hargreaves \\
a \\
Kimberley-Penman = Penman \\
a, b \\
Jensen-Haise \\
Cr, tx \\
Penman-Monteith \\
rs, ra \\
Oudin \\
k1, k2 \\
FAO-pm-reference + crop coefficient \\
crop coefficient($k_max$) 
\section{Evaluation of PET models with original constant values/results}
\section{Evaluation of PET models with calibrated constant values/results}



\subsubsection{Inputs}

$R_n$ net longwave radiation
\begin{equation}
R_nl = f_{cd} (a_1+b_1\sqrt{e_a})\sigma T^4
\end{equation}
if 24-hour or longer time steps...$T^4$ transforms to ($T^4_{max}-T^4_{min}$)/2. \\
$\sigma$ is for daily values 4.901 x 10$^{-9}$ MJm$^{-2}$d$^{-1}$K$^{-4}$ with Rnl in MJm$^{-2}$d$^{-1}$ \\
for hourly calculations $\sigma$ = 2.042 x 10$^{-10}$ MJm2 h$^{-1}$K$^{-4}$, Rnl is in MJm$^{-2}$ h$^{-1}$.

Wright and Jensen \cite{wright1972peak} developed an expression for $f_{cd}$:
\begin{equation}
f_{cd} = a \frac{R_S}{R_{SO}} + b
\end{equation}
a and b are empirical coefficients. General a= 1.3, b=0.3, $a_1$ = 0.39 and $b_1$=0.158

Bulleted lists look like this:
\begin{itemize}[leftmargin=*,labelsep=5.8mm]
\item	First bullet
\item	Second bullet
\item	Third bullet
\end{itemize}

Numbered lists can be added as follows:
\begin{enumerate}[leftmargin=*,labelsep=4.9mm]
\item	First item 
\item	Second item
\item	Third item
\end{enumerate}

The text continues here.

\subsection{Figures, Tables and Schemes}

All figures and tables should be cited in the main text as Figure 1, Table 1, etc.

\begin{figure}[H]
\centering
\includegraphics[width=2 cm]{Definitions/logo-mdpi}
\caption{This is a figure, Schemes follow the same formatting. If there are multiple panels, they should be listed as: (\textbf{a}) Description of what is contained in the first panel. (\textbf{b}) Description of what is contained in the second panel. Figures should be placed in the main text near to the first time they are cited. A caption on a single line should be centered.}
\end{figure}   
 
Text

Text

\begin{table}[H]
\caption{This is a table caption. Tables should be placed in the main text near to the first time they are cited.}
\centering
%% \tablesize{} %% You can specify the fontsize here, e.g., \tablesize{\footnotesize}. If commented out \small will be used.
\begin{tabular}{ccc}
\toprule
\textbf{Title 1}	& \textbf{Title 2}	& \textbf{Title 3}\\
\midrule
entry 1		& data			& data\\
entry 2		& data			& data\\
\bottomrule
\end{tabular}
\end{table}

Text

Text

%\begin{listing}[H]
%\caption{Title of the listing}
%\rule{\textwidth}{1pt}
%\raggedright Text of the listing. In font size footnotesize, small, or normalsize. Preferred format: left aligned and single spaced. Preferred border format: top border line and bottom border line.
%\rule{\textwidth}{1pt}
%\end{listing}


\subsection{Formatting of Mathematical Components}

This is an example of an equation:

\begin{equation}
a + b = c
\end{equation}
%% If the documentclass option "submit" is chosen, please insert a blank line before and after any math environment (equation and eqnarray environments). This ensures correct linenumbering. The blank line should be removed when the documentclass option is changed to "accept" because the text following an equation should not be a new paragraph. 

Please punctuate equations as regular text. Theorem-type environments (including propositions, lemmas, corollaries etc.) can be formatted as follows:
%% Example of a theorem:
\begin{Theorem}
Example text of a theorem.
\end{Theorem}

The text continues here. Proofs must be formatted as follows:

%% Example of a proof:
\begin{proof}[Proof of Theorem 1]
Text of the proof. Note that the phrase `of Theorem 1' is optional if it is clear which theorem is being referred to.
\end{proof}
The text continues here.

%%%%%%%%%%%%%%%%%%%%%%%%%%%%%%%%%%%%%%%%%%
\section{Discussion}

Authors should discuss the results and how they can be interpreted in perspective of previous studies and of the working hypotheses. The findings and their implications should be discussed in the broadest context possible. Future research directions may also be highlighted.

%%%%%%%%%%%%%%%%%%%%%%%%%%%%%%%%%%%%%%%%%%
\section{Materials and Methods}

Materials and Methods should be described with sufficient details to allow others to replicate and build on published results. Please note that publication of your manuscript implicates that you must make all materials, data, computer code, and protocols associated with the publication available to readers. Please disclose at the submission stage any restrictions on the availability of materials or information. New methods and protocols should be described in detail while well-established methods can be briefly described and appropriately cited.

Research manuscripts reporting large datasets that are deposited in a publicly available database should specify where the data have been deposited and provide the relevant accession numbers. If the accession numbers have not yet been obtained at the time of submission, please state that they will be provided during review. They must be provided prior to publication.

Interventionary studies involving animals or humans, and other studies require ethical approval must list the authority that provided approval and the corresponding ethical approval code. 

%%%%%%%%%%%%%%%%%%%%%%%%%%%%%%%%%%%%%%%%%%
\section{Conclusions}

This section is not mandatory, but can be added to the manuscript if the discussion is unusually long or complex.

%%%%%%%%%%%%%%%%%%%%%%%%%%%%%%%%%%%%%%%%%%
\section{Patents}
This section is not mandatory, but may be added if there are patents resulting from the work reported in this manuscript.

%%%%%%%%%%%%%%%%%%%%%%%%%%%%%%%%%%%%%%%%%%
\vspace{6pt} 

%%%%%%%%%%%%%%%%%%%%%%%%%%%%%%%%%%%%%%%%%%
%% optional
%\supplementary{The following are available online at \linksupplementary{s1}, Figure S1: title, Table S1: title, Video S1: title.}

% Only for the journal Methods and Protocols:
% If you wish to submit a video article, please do so with any other supplementary material.
% \supplementary{The following are available at \linksupplementary{s1}, Figure S1: title, Table S1: title, Video S1: title. A supporting video article is available at doi: link.}

%%%%%%%%%%%%%%%%%%%%%%%%%%%%%%%%%%%%%%%%%%
\authorcontributions{For research articles with several authors, a short paragraph specifying their individual contributions must be provided. The following statements should be used ``conceptualization, X.X. and Y.Y.; methodology, X.X.; software, X.X.; validation, X.X., Y.Y. and Z.Z.; formal analysis, X.X.; investigation, X.X.; resources, X.X.; data curation, X.X.; writing--original draft preparation, X.X.; writing--review and editing, X.X.; visualization, X.X.; supervision, X.X.; project administration, X.X.; funding acquisition, Y.Y.'', please turn to the  \href{http://img.mdpi.org/data/contributor-role-instruction.pdf}{CRediT taxonomy} for the term explanation. Authorship must be limited to those who have contributed substantially to the work reported.}

%%%%%%%%%%%%%%%%%%%%%%%%%%%%%%%%%%%%%%%%%%
\funding{Please add: ``This research received no external funding'' or ``This research was funded by NAME OF FUNDER grant number XXX.'' and  and ``The APC was funded by XXX''. Check carefully that the details given are accurate and use the standard spelling of funding agency names at \url{https://search.crossref.org/funding}, any errors may affect your future funding.}

%%%%%%%%%%%%%%%%%%%%%%%%%%%%%%%%%%%%%%%%%%
\acknowledgments{In this section you can acknowledge any support given which is not covered by the author contribution or funding sections. This may include administrative and technical support, or donations in kind (e.g., materials used for experiments).}

%%%%%%%%%%%%%%%%%%%%%%%%%%%%%%%%%%%%%%%%%%
\conflictsofinterest{Declare conflicts of interest or state ``The authors declare no conflict of interest.'' Authors must identify and declare any personal circumstances or interest that may be perceived as inappropriately influencing the representation or interpretation of reported research results. Any role of the funders in the design of the study; in the collection, analyses or interpretation of data; in the writing of the manuscript, or in the decision to publish the results must be declared in this section. If there is no role, please state ``The funders had no role in the design of the study; in the collection, analyses, or interpretation of data; in the writing of the manuscript, or in the decision to publish the results''.} 

%%%%%%%%%%%%%%%%%%%%%%%%%%%%%%%%%%%%%%%%%%
%% optional
\abbreviations{The following abbreviations are used in this manuscript:\\

\noindent 
\begin{tabular}{@{}ll}
MDPI & Multidisciplinary Digital Publishing Institute\\
DOAJ & Directory of open access journals\\
TLA & Three letter acronym\\
LD & linear dichroism
\end{tabular}}

%%%%%%%%%%%%%%%%%%%%%%%%%%%%%%%%%%%%%%%%%%
%% optional
\appendixtitles{no} %Leave argument "no" if all appendix headings stay EMPTY (then no dot is printed after "Appendix A"). If the appendix sections contain a heading then change the argument to "yes".
\appendix
\section{}
\unskip
\subsection{}
The appendix is an optional section that can contain details and data supplemental to the main text. For example, explanations of experimental details that would disrupt the flow of the main text, but nonetheless remain crucial to understanding and reproducing the research shown; figures of replicates for experiments of which representative data is shown in the main text can be added here if brief, or as Supplementary data. Mathematical proofs of results not central to the paper can be added as an appendix.

\section{}
All appendix sections must be cited in the main text. In the appendixes, Figures, Tables, etc. should be labeled starting with `A', e.g., Figure A1, Figure A2, etc. 

%%%%%%%%%%%%%%%%%%%%%%%%%%%%%%%%%%%%%%%%%%
%\reftitle{References}

% Please provide either the correct journal abbreviation (e.g. according to the “List of Title Word Abbreviations” http://www.issn.org/services/online-services/access-to-the-ltwa/) or the full name of the journal.
% Citations and References in Supplementary files are permitted provided that they also appear in the reference list here. 

%=====================================
% References, variant A: external bibliography
%=========================
\externalbibliography{yes}
\bibliography{evapo_article}


% The following MDPI journals use author-date citation: Arts, Econometrics, Economies, Genealogy, Humanities, IJFS, JRFM, Laws, Religions, Risks, Social Sciences. For those journals, please follow the formatting guidelines on http://www.mdpi.com/authors/references
% To cite two works by the same author: \citeauthor{ref-journal-1a} (\citeyear{ref-journal-1a}, \citeyear{ref-journal-1b}). This produces: Whittaker (1967, 1975)
% To cite two works by the same author with specific pages: \citeauthor{ref-journal-3a} (\citeyear{ref-journal-3a}, p. 328; \citeyear{ref-journal-3b}, p.475). This produces: Wong (1999, p. 328; 2000, p. 475)


%%%%%%%%%%%%%%%%%%%%%%%%%%%%%%%%%%%%%%%%%%
%% optional
\sampleavailability{Samples of the compounds ...... are available from the authors.}

%% for journal Sci
%\reviewreports{\\
%Reviewer 1 comments and authors’ response\\
%Reviewer 2 comments and authors’ response\\
%Reviewer 3 comments and authors’ response
%}

%%%%%%%%%%%%%%%%%%%%%%%%%%%%%%%%%%%%%%%%%%
\end{document}

